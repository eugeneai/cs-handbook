\documentclass[14pt, openany, twoside, draft]{extbook} % Computer Modern font calls
% final
%\usepackage[usenames]{color}
%\usepackage{fancybox}
%\usepackage{algorithmic} % noend
%\usepackage[boxed]{algorithm} % boxed, ruled, plain (also)
%\usepackage{longtable}%длинные таблицы

\usepackage[final]{graphicx}
\usepackage{algorithm}

% Main style definition
% ---------------------
% ISU standard handbook
%\usepackage[fontdir="",handbook,fancybot,times,inconsolata,smalltitles,microtyping]{subook}
\usepackage[handbook,liberation,fancybot,inconsolata,smalltitles,microtyping]{subook}
\usepackage{xcolor}

\colorlet{pcolor}{blue}
\colorlet{fcolor}{red}
\newcommand{\e}[2][fcolor]{\textcolor{pcolor}{[}\textcolor{#1}{#2}\textcolor{pcolor}{]}}

% ISU standard monograph (less restrictive and more artistic)
% \usepackage[monograph,mag,times,smalltitles,fancybot,listbib,ptfonts,microtyping]{subook}

% Some artistism for monograph
% ----------------------------
%\makeatletter{}
%\renewcommand\su@chapter@font{\sffamily\sfcpshape\bfseries}
%\renewcommand\su@chapter@font@size{\LARGE}
%\makeatother{}
%\floatname{algorithm}{Процедура}
%\renewcommand{\listalgorithmname}{Список процедур}
%\renewcommand\cftsecnumwidth{5ex}
%\tolerance=5000
%\renewcommand{\chaptername}{Глава}



\usepackage{tikz}
\usetikzlibrary{arrows,arrows.meta,shapes}

\long\def\rem#1{}
\def\emphbib#1{#1}
\newenvironment{questions}{\subsubsection*{Вопросы для самопроверки}\begin{enumerate}\itemsep0pt minus 0.3pt\parskip0pt plus 0.3pt}{\end{enumerate}}

\newtheorem{example}{Пример}[chapter]
\hypersetup{
    bookmarks=true,         % show bookmarks bar?
    unicode=true,           % non-Latin characters in Acrobat’s bookmarks
    pdftoolbar=true,        % show Acrobat’s toolbar?
    pdfmenubar=true,        % show Acrobat’s menu?
    pdffitwindow=false,     % window fit to page when opened
    pdfstartview={FitH},    % fits the width of the page to the window
    pdftitle={Рекурсивно-Логическое Программирование},    % title
    pdfauthor={Евгений Александрович Черкашин},     % author
    pdfsubject={Методическое пособие},   % subject of the document
    pdfcreator={LaTeX},   % creator of the document
    pdfproducer={LaTeX}, % producer of the document
    pdfkeywords={Пролог} {Искусственный интеллект} {Планирование действий}, % list of keywords
    pdfnewwindow=true,      % links in new window
    colorlinks=true,       % false: boxed links; true: colored links
    linkcolor=[rgb]{0 0.4 0.1},          % color of internal links (black)
    citecolor=blue,        % color of links to bibliography
    filecolor=black,      % color of file links
    urlcolor=[rgb]{0.3 0.0 0.3}           % color of external links
}

%\renewcommand{\headrulewidth}{1pt}

\clubpenalty=3000
\widowpenalty=3000
%\brokenpenalty=10000
%\floatingpenalty=10000

%% \setdefaultlanguage{russian}
%% \setmainlanguage{russian}
%% \setotherlanguage{english}

\renewcommand\baselinestretch{1.1}
\parskip=0pt plus 0.3pt
\begin{document}
% \itemsep3pt plus 0pt minus 3pt
% \widowpenalty=10000
% \clubpenalty=10000
% \renewcommand\sutitlefontface{\Large\ptsans\nwshape\bfseries}
% \theorembodyfont{\rmfamily}

\lstset{language=Prolog, morecomment=[l]{\%}}
\renewcommand{\chaptername}{} % for ISU Handbooks
\renewcommand{\refname}{Рекомендуемая литература} % ... also
\renewcommand{\bibname}{\refname}
\begin{titlepage}
\thispagestyle{empty}
\begin{center}{\small{}
Министерство образования и науки
Российской федерации \\
Федеральное государственное бюджетное образовательное\\
учреждение высшего профессионального образования\\
<<Иркутский государственный университет>>\\[2ex]
Федеральное государственное бюджетное образовательное\\
учреждение высшего профессионального образования\\
Научный исследовательский Иркутский государственный технический университет>>\\[2ex]
    Учреждение Российской академии наук \\
<<Институт динамики систем и теории управления \\
Сибирского отделения РАН>>
}
\vfill
\hbox to \linewidth{\hfill\bfseries Е.~А.~Черкашин, В.~О.~Обризан\hfill}
 \vspace{2em}
{\large\bfseries Технологии повышения производительности вычислительных систем}\\
 \vspace{2em}
{Учебное пособие}
\vfill
%\vfill
\vfill
 \textbf{Иркутск 2015}
\end{center}
\end{titlepage}
%\newpage
%\begingroup
%\normalfont \ttfamily Это \textbf{текст} в \itshape нормальном \bfseries фонте\sffamily етноф\nwshape фонт\rmcpshape тноф\sfcpshape фонт\ttfamily тноф...
%\endgroup

\newpage
\begin{mygroup}
\thispagestyle{empty}
\noindent УДК 681.3.06 (075.8)\\ % (075.8) = Handbook
% \noindent УДК~165:681.3.06+62-52\\

% Программирование для ЭВМ--Учебники и пособия для вузов
\noindent ББК 32.973-01я73\\ % я73 - Учебник для вузов -01=

% 86.4 = Логика
% 32.973 = Программное обеспечение
% 32.813 = Искусственный интеллект
% -01 =
% ББК 87.4:32.973-01+32.813\\ % monograph
\noindent\mbox{}\hspace{2em}Ч-48 % Author Sign (Авторский знак)

\begin{center}\small
\e{Печатается по решению ученого совета ИМЭИ\\[2ex]
\bfseries Издание выходит в рамках Программы\\
стратегического развития ФГБОУ ВПО <<ИГУ>>\\
на 2012--2016 гг., проект Р121-02-001}
\end{center}
\vspace{1ex}
\begin{center}\small
\textbf{Рецензенты:} \\
\e{канд.~техн.~наук~{\em },\\ канд.~физ.-мат.~наук~{\em }}
\end{center}
\vfill
\noindent\begin{minipage}[t]{2em}
\noindent\mbox{}\\
Ч-48
\end{minipage}%
\begin{minipage}[t]{0.95\linewidth}
\setlength{\parindent}{5ex}
\noindent{\bfseries Черкашин~Е.~А.}

Технологии повышения производительности вычислительных систем\,{}:
учеб.~пособие\,/~Е.~А.~Черкашин, Обризан~В.~И.~--
Иркутск\,: Изд-во ИГУ, 2015.~-- \pageref{lastpage}~c.

{\bfseries \e{ISBN 978-5-9624-ХХХ-Х}}
\vspace{2ex}

\begingroup\small\parskip0pt
\vspace{1ex} В пособии представлены материалы лекций и задания
лабораторных работ по курсу <<Вычислительные системы>>.  Задача курса
состоит в развитии навыков эффективного программирования, включающего
как приобретение опыта использования специализированных структур
данных, инструментов оценки и повышения производительности
компилируемых программ, а также методов агрегирования вычислительных
ресурсов и метакомпьютинга.  Пособие содержит задания на лабораторный
практикум по темам <<\e{Измерение производительности}>>,
<<\e{Эффективные структуры данных}>>, <<\e{Оптимизация программ}>>,
<<\e{Программы-химеры}>> и <<\e{Параллельное и распределенное
  программирование}>>.

Предполагаемая аудитория пособия: студенты и молодые специалисты
инженерных специальностей, связанных с программированием
вычислительных систем, серверов, а также встроенных систем.

\mbox{}
\endgroup
\end{minipage}
\mbox{}\hspace{0.7\linewidth}
\begin{minipage}{0.3\linewidth}\small
\noindent УДК 681.3.06 (075.8)\\
\noindent ББК 32.973-01я73
\end{minipage}

\vfill
\noindent\begin{minipage}[t]{0.35\linewidth}\small
\noindent \e{ISBN 978-5-9624-ХХХХ-Х}
\end{minipage}%
\begin{minipage}[t]{0.65\linewidth}\small
\begin{itemize}
\setlength{\itemsep}{-0.5ex}
\setlength{\parsep}{0pt}
\item[\copyright{}] Черкашин~Е.~А., Обризан~В.И. 2015
\item[\copyright{}] ФГБОУ ВПО <<ИГУ>>, 2015
\item[\copyright{}] ФГБОУ ВПО НИ ИрГУ, 2015
%\item[\copyright{}] ФГБОУ ВПО НИ ИрГТУ, 2014 % TODO: Check the name
\item[\copyright{}] Институт динамики систем и теории управления СО РАН, 2015
\end{itemize}
\end{minipage}
\end{mygroup}
\clearpage
%\setcounter{page}{2}
\tableofcontents
\clearpage

\newpage
\chapter*{Предисловие}

% Вашему вниманию предлагается учебное пособие к курсу «Введение в
% оптимизацию производительности программного обеспечения». В книге
% рассматриваются актуальные вопросы эффективного использования
% многоядерных процессоров, которые прочно занимают рынок
% микропроцессоров. Также рассмотрены вопросы, касающиеся организации
% процесса тестирования производительности, устройство современных
% микроархитектур, библиотеки многопоточного программирования, вопросы
% оптимизации последовательных программ.

Решение вопросов повышения производительности программного обеспечения
--- один из основных видов деятельности профессионального
программиста.  Важно, не только, чтобы программа работала корректно,
выдавала тот результат, который от нее ожидают, но и делала это за
приемлемое для масштаба решаемой задачи время.  Для решения этой
задачи используются различные методы и технологии, включающие как
формальный анализ алгоритма программы \cite{progproof}, реализацию
специальных структур данных на определенных этапах решения
вычислительной задачи \cite{algstruct}, так и настройку программного
кода на особенности архитектуры вычислительной системы.  Формальный
анализ позволяет критически оценить качество алгоритмов, в том числе,
условия его корректного использования, возможности обобщения на другие
структуры и типы данных, зависимость скорости выполнения и
потребляемых ресурсов от размера обрабатываемых данных.  Инженерный
аспект программирования проявляется при реализации программного кода.
Например, индексы реляционных баз данных на сервере используют
древовидные структуры B*-деревья для обеспечения эффективного
отображения ключевого значения на физический номер соответствующей
записи.  При вычислении произведения матриц целых чисел некоторое
повышение производительности достигается при использовании векторных
инструкций процессора Intel вместо последовательного выполнения
отдельных операций умножения и сложения друг за другом.

Материал \e{пособия} представляет в обзорной форме теорию и
практические (инженерные) аспекты решения задач улучшения алгоритмов и
программного кода.  Материал разделен на пять частей, посвященные,
соответственно, методикам оценки и измерения производительности
\e{алгоритмов и} программ; представлению \e{классических} структур
данных; вопросам настройки программного кода на использования
имеющихся вычислительных средств микропроцессора; программным
технологиям эффективного программирования приложений, основывающихся
на сопряжении различных сред программирования; реализации программ на
кластерных вычислительных архитектурах.  В каждой части приведен
перечень лабораторных работ, качественное выполнение которых \e{как
  всегда} позволяет закрепить излагаемый материал.

Получаемые в данном курсе знания и навыки необходимо сразу же начинать
использовать в своей практической деятельности, т.е. в ближайшем
курсовом, дипломном проекте или магистерской диссертации
(квалификационной работы).  Вопросам оптимизации алгоритмов и программ
в тексте квалификационной работы следует посвятить раздел или даже
главу, если тема работы посвящена непосредственно разработке
алгоритма.  На практике всякий раз как реализуется новая процедура
необходимо предварительно оценить возможности использования методик
улучшения производительности и производить дальнейшее кодирование с
учетом этой оценки.  Если уже была реализован некоторый вычислительный
процесс, то перед внедрением оптимизации следует обязательно сохранить
неоптимизированную версию процедуры, так как эта версия, вероятно,
будет значительно понятнее программисту, изучающему ваш код, чем
улучшенная.  Две версии одной и той же процедуры можно хранить в
системе контроля версий, например, Subversion \cite{svn} или GIT
\cite{git}, в разных ветках (branches).

Книга основана на опыте преподавания дисциплин «Многоядерное
программирование», «Введение в оптимизацию производительности
программного обеспечения», которые авторы вели в Харьковском
национальном университете радиоэлектроники с 2007~г., а также курса
<<Вычислительные системы>> в Институте кибернетики Иркутского
государственного технического университета с 2012~г.

Авторы пособия являются приверженцем открытых технологий, свободных
книг, научного метода познания мира и открытого программного
обеспечения.  Адрес исходного кода методического пособия~---
\url{https://github.com/eugeneai/cs-handbook.git}. Исхоный код
разрешено использовать в соответствии с лицензией
\foreignlanguage{english}{GNU Free Documentation License (GNU FDL)},
которая предлагает любому широкие права на текст методического
пособия: лицензия допускает воспроизведение, распространение и
изменение исходного текста пособия, в том числе и в коммерческих
целях.  Со своей стороны лицензиат обязуется соблюдать условия
лицензии.  Эти условия включают в себя, в частности, обязательное
указание имени авторов оригинального текста \cite{GNUFDL}.  Авторы
пособия были бы благодарны, если в производных работах на основе
данного методического пособия указывались их имена.  В случае
невозможности сделать это по какой-либо причине, пожалуйста свяжитесь
с нами по электронной почте.

Лицензия GNU FDL, будучи основанная на концепции так называемого
``копилефта'', требует, чтобы любые копии методического пособия (в том
числе производные работы) распространялись на тех же самых условиях,
без добавления дополнительных ограничений.  Каждая копия должна
сопровождаться ссылкой в электронном виде на текст лицензии.  Адрес текста лицензии~---  \url{http://www.gnu.org/licenses/fdl.html}.

\e{
В тексте пособия использована следующая разметка:
\begin{description}
\item[Жирным шрифтом] выделяются имена существительные и глаголы, на которые, по мнению автора, следует обратить внимание.
\item[\normalfont{\tt Моноширинным шрифтом}] приводятся программы, отрывки программ в основном тексте пособия, а также имена идентификаторов, т.~е. все, что имеет какое"=либо отношение к {\bf тексту программы}.
\item[\normalfont{\em Наклонным шрифтом}] выделяются {\bf новые} термины, вводимые в текст и возникающие, например, в определениях, а также текст выделенных примеров.
\item[\normalfont При помощи <<кавычек>>] выделяются метафоры, значения, элементы текстов программ, цитаты, слова, использованные в переносном смысле, и т.~д.
%\item[\normalfont{\sf Рубленым шрифтом}] декорируются тексты, которые надо как-то особо выделить на общем фоне.
\end{description}
}

\medskip
\noindent\hbox to \linewidth{\hfill\sf Старший~научный сотрудник ИДСТУ СО РАН,}
\noindent\hbox to \linewidth{\hfill\sf доцент кафедры ИТ ИМЭИ ИГУ}
\noindent\hbox to \linewidth{\hfill\sf кандидат~технических~наук}
\noindent\hbox to \linewidth{\hfill\sf Е.~А.~Черкашин}

\vfill
\makeatletter
\noindent{\sf P.~S.} Авторы будут рады получить отзывы на данную
работу по электронной почте
\href{mailto:eugeneai@icc.ru}{\tt{}eugeneai@icc.ru}, в поле <<{\tt
  тема}>> просим указывать <<CS-2015>>.
\makeatother

\chapter{Изучение программы}

Быстрее и удобней всего изучать программу (как чужую, так и свою),
начиная с общего описания ее функционирования, формируемого
комментариями и документацией, затем исследуются основные структуры
данных, упомянутые в текстовом описании.  Следующий этап --- это
ознакомление с интерфейсом прикладного
программирования (API\footnote{Англ. -- Application programming
  interface.}).  Пример хорошо оформленного модуля программы
приведен в статье \cite{fogel2009}.  Затем уже можно переходить к
изучению реализации функций API программы.

Чтение программного кода позволяет оценить реализацию программы в
статическом структурном аспекте, исполнение алгоритма над конкретными
данными анализируется при помощи инструментов отладки кода.  К таким
инструментам относятся средства среды программирования, при помощи
которых в программный код можно вносить операторы печати промежуточных
значений переменных; программные отладчики, позволяющие проследить
выполнение программы непосредственно на процессоре и в конкретной
среде исполнения\footnote{Англ. Runtime environment.}; различные
средства измерения корректности и производительности программы.  К
последнему классу, например, относятся библиотеки трассировки функций
управления динамической памятью (Wargrind) и профилировщики (GNU
gprof).  Профилировщики запускают программу с конкретными данными и
считают количество обращений к той или иной функции.

Профилирование программ рассмотри чуть позже в разделе \ref{sec:prof},
а сейчас рассмотрим возможности открытого отладчика GNU GDB.

\section{Основные функции отладчика}

Задача отладчика GDB состоит в том, чтобы предоставить средства
отслеживания процесса выполнения программы и возможность узнать, что
программа делала и каковы были значения переменных в момент ее
аварийной остановки.  GDB запускает программу, устанавливает ее среду
исполнения, приостанавливает исполнение программы при выполнении
заданных условий, отображает состояние регистров процессора и
переменных, менять значения переменных, предоставляя возможность
скорректировать небольшую ошибку и перейти к изучению новой.

GDB применяется для отладки скомпилированных программ на языках C,
C++, D, Fortran, Pascal, Ada и др.  В таких языках как D и Pascal не
все возможности GDB могут быть задействованы в процессе отладки,
языкам C и C++ предоставляется полный инструментарий GDB.

\subsection{И т.п.}


\begin{questions}
\item{} Перечислите .....
\item{} Дайте характеристику ...
\item{} В чем суть ....
\item{} Приведите ...
\item{} Какие ...
\end{questions}

\chapter{Оптимизация структур данных}

\chapter{Комбинирование сред программирования}

\emph{Средой программирования} будем называть следующую
комбинацию сервисов, предоставляемых тем или иным \e{инструментарием} ...

\chapter{Адаптация программ к аппаратному обеспечению}

\chapter{Реализация параллельных схем алгоритмов \e{на кластерах}}


\chapter*{Заключение}


Изучение ...

В книге ...


%\listoffigures
%\addcontentsline{toc}{section}{Список иллюстраций}
%\listoftables
%\addcontentsline{toc}{section}{Список таблиц}
\begin{thebibliography}{99}\itemsep1pt \parskip 0pt plus 0.3pt
\bibitem{Anderson} Андерсон~Р. \emphbib{Доказательство правильности программ}\,{}: пер. с англ.\,{}/ Р.~Андерсон. -- М.\,:\,Мир, 1982. -- 168~c.: ил.
\bibitem{Bratko} Братко~И. \emphbib{\href{http://royallib.ru/book/bratko_ivan/programmirovanie_na_yazike_prolog_dlya_iskusstvennogo_intellekta.html}{Программирование на языке ПРОЛОГ для искусственного интеллекта}}\,{}: пер. с англ.\,/ И.~Братко. -- М.\,:~Мир, 1990. -- 560~c.: ил.
\bibitem{Vass:2000} Васильев~С.~Н. \emphbib{\href{http://bookfi.org/book/616050}{Интеллектное управление динамическими системами}}\,{}/ С.~Н.~Васильев, А.~К.~Жерлов, Е.~А.~Федосов, Б.~Е.~Федунов. -- М.\,:~Физматлит, 2000. -- 352~с: ил.
\bibitem {AIDictionary} \emphbib{\href{http://aihandbook.intsys.org.ru/index.php/intro/ai-handbook}{Искусственный интеллект\,{}: в 3~кн.}}\,{}/ под ред. Э.~В. Попова. -- М.\,:~Радио и связь, 1990. -- 464 c.:\,{}ил.
\bibitem{Lauriere} Лорьер.~Ж.-Л.  \emphbib{\href{http://publ.lib.ru/ARCHIVES/L/LOR'ER_Jan_Lui/_Lor'er_J.L..html}{Системы искусственного интеллекта}\,{}: пер. с франц.}\,{}/ Ж.-Л. Лорьер. -- М.\,:~Мир, 1991. -- 568~с.: ил.
\bibitem{Malpas} Малпас~Дж. \emphbib{\href{http://padaread.com/?book=40731&pg=1}{Реляционный язык Пролог и его применение}}\,{}/ Дж.~Малпас. -- М.\,:~Наука, 1990. -- 464~с.
\bibitem{math_slov:88} \emphbib{\href{https://app.box.com/shared/793ukgvblxmj0hh6btw4}{Математический энциклопедический словарь}}\,{}/ гл.~ред. Ю.~В.~Прохоров. -- М.\,:~Сов.~энциклопедия, 1988. -- 847~c.
\bibitem{DDW} Непейвода~Н.~Н. \emphbib{\href{http://www.logic-books.info/taxonomy/term/215}{Прикладная логика\,{}: учеб. пособие}}\,{}/ Н.~Н.~Непейвода. -- 2-е изд. -- Новосибирск\,{}:~Изд-во Новосиб. ун-та, 2000. -- 521~c.: ил.
\bibitem{DDWII} Непейвода~Н.~Н.  \emphbib{\href{http://philosophy.ru/library/logic_math/library/nepeivoda_prog.pdf}{Основания программирования}}\,{}/ Н.~Н.~Непейвода, И.~Н.~Скопин. -- Москва; Ижевск\,{}:~Институт компьютерных исследований, 2003 -- 880~c.: ил.
\bibitem {Russell} Рассел~С. \href{http://www.aiportal.ru/downloads/books/ai-modern-approach-2-edition-by-rassel-norvig.html}{Искусственный интеллект: современный подход}\,{}: пер. с англ.\,{}/ С.~Рассел, П.~Новриг. 2-е изд. -- М.\,:~Изд. дом <<Вильямс>>, 2006. -- 1408~c.: ил.
\bibitem{WIKI-DCG} \emphbib{\href{https://en.wikipedia.org/wiki/Definite_clause_grammar}{DC-грамматика}} [Электронный ресурс]\,{}// Wikipedia, The Free Encyclopedia\,{}: сайт. -- URL:\texttt{https://en.wikipedia.org/wiki/Definite\_clause\linebreak\_grammar}. (дата обращения: 28.11.2013).
\bibitem{GNUP} \emphbib{\href{http://www.gprolog.org/}{The GNU Prolog web site [Электронный ресурс]\,{}: сайт}}. URL:\url{http://www.gprolog.org/}. (дата обращения: 28.11.2013).
\bibitem{SWIP} \emphbib{\href{http://www.swi-prolog.org/}{SWI-Prolog's
      home [Электронный ресурс]\,{}:
      сайт}}. URL:\url{http://www.swi-prolog.org/}. (дата обращения:
  28.11.2013).
\bibitem{fogel2009} К.~Фегель. Дельта-редактор Subversion: Интерфейс и
  онтология. // Идеальный код. / Под редакцией Э.~Орама и Г.Уилсона --
  СПб.\,: Питер, 2009. -- С.~27--45.
\end{thebibliography}
\label{lastpage}
\newpage
\thispagestyle{empty}
\mbox{}

\vfill\vfill\vfill\vfill

\hfill{}{\small\itshape Учебное издание}
\vspace{4ex}
\begin{center}
{\small\textbf{Черкашин} Евгений Александрович\\[1em]}
{\bfseries Рекурсивно"=логическое программирование}\\[1em]
ISBN~978-5-9624-0938-2
\vfill

\small
Редактор \textit{Г.~А.~Борисова}\\
Верстка \textit{Е.~А.~Черкашин}

\vfill{}
{\small Макет подготовлен при помощи системы \LuaLaTeX\\\mbox{}}
\vfill{}

Темплан 2013\,{}г. Поз.\,{}186

\end{center}
\begin{center}\small
\noindent Подписано в печать 28.12.2013.
Формат~60$\times$90 1/16.\\  %Гарнитура \sutypeface{}.
%Верстка \LuaLaTeXe.
%Бумага офсетная. Печать офсетная. Усл.печ.л.
Уч.-изд.\,{}л.\,{}6,4. Усл.\,{}печ.\,{}л. 6,8. Тираж~100~экз. Заказ~170
\end{center}
\vspace{1           ex}
\begin{center}\small
Издательство ИГУ\\{}
664003, г.\,{}Иркутск, бульвар Гагарина, 36
\end{center}
\end{document}
%%%%%%%%%%%%%%%%%%%%%%%%%%%%%%%%%%%%%%%%%%%%%%%%%%%
%%% Local Variables:
%%% TeX-engine: luatex
%%% End:
